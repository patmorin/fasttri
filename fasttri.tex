\documentclass[kpfonts]{patmorin}
\listfiles
\usepackage{pat}
\usepackage{paralist}
\usepackage{dsfont}  % for \mathds{A}
\usepackage[utf8]{inputenc}

\usepackage{graphicx}
\usepackage[noend]{algorithmic}

\usepackage{xcolor}
\definecolor{light-gray}{gray}{0.95}

\usepackage[normalem]{ulem}
\usepackage{cancel}
\usepackage{enumitem}


\newcommand{\snote}[1]{\fcolorbox{red}{yellow}{#1}}
\newcommand{\pnote}[1]{\ \newline\noindent\fcolorbox{red}{yellow}{\begin{minipage}{\textwidth}#1\end{minipage}}}
\setlength{\parskip}{1ex}

\DeclareMathOperator{\A}{\mathds{A}}
\DeclareMathOperator{\sn}{sn}
\DeclareMathOperator{\qn}{qn}

\renewcommand{\SS}{\mathcal{S}}

\newcommand{\Oh}{\mathcal{O}}


%Piotreks overloads
\let\le\leqslant
\let\ge\geqslant
\let\leq\leqslant
\let\geq\geqslant
\let\nleq\nleqslant
\let\ngeq\ngeqslant
%%Piotrek end

\newcommand{\aref}[1]{(X\ref{a:#1})}
\newcommand{\alabel}[1]{\label{a:#1}}

\newcommand{\itemref}[1]{(\ref{#1})}

\title{\MakeUppercase{A Fast Algorithm for Layered $H$-Partitions of Planar Graphs}}
\author{
  Pat Morin%
    \thanks{School of Computer Science, Carleton University, Canada. This research was partially supported by NSERC.}
}
% \date{}

\begin{document}
\maketitle

\begin{abstract}
  Dujmović \etal\  (FOCS2019) recently proved that every planar graph $G$ is a subgraph of $H\boxtimes P$, where $\boxtimes$ denotes the strong graph product, $H$ is a graph of treewidth 8 and $P$ is a path.  This result has found numerous applications to linear graph layouts, graph colouring, and graph labelling.  The proof given by Dujmović \etal\  is based on a similar decomposition of Pilipczuk and Siebertz (SODA2019) which is constructive and leads to an $O(n^2)$ time algorithm for finding $H$ and the mapping from $V(G)$ onto $V(H\boxtimes P)$.  In this note, we show that this algorithm can be made to run in $O(n\log n)$ time.
\end{abstract}

\section{Introduction}

The \emph{strong product} $G_1\boxtimes G_2$ of two graphs $G_1$ and $G_2$ is the graph whose vertex set is the Cartesian product $V(G_1)\times V(G_2)$ in which the vertices $(v,x)$ and $(w,y)$ are adjacent if and only if
\begin{compactitem}
  \item $v=w$ and $xy\in E(G_2)$;
  \item $vw\in E(G_1)$ and $x=y$; or
  \item $vw\in E(G_1)$ and $xy\in E(G_2)$.
\end{compactitem}
Dujmović \etal\  \cite{dujmovic.joret.ea:planar} recently proved the following \emph{product structure theorem} for planar graphs:

\begin{thm}[Dujmović \etal\ 2019]\thmlabel{product-structure}
  For any $n$-vertex planar graph $G$, there exists a graph $H$ of treewidth at most 8 and a path $P$ such that $G$ is a subgraph of $G^+:=H\boxtimes P$.
\end{thm}

Though still very new, \thmref{product-structure} has been used to solve a number of longstanding open problems on planar graphs:
\begin{itemize}
  \item \thmref{product-structure} has been used to show that the queue-number of every planar graph is upper bounded by a constant.  This solves an open problem of Heath, Leighton, and Rosenberg posed in 1992 \cite{heath.leighton.ea:comparing}.
  \item \thmref{product-structure} has been used to show that the nonrepetitive chromatic number of every planar graph is upper bounded by a constant. This solves an open problem of Alon \etal\  \cite{alon.grytczuk.ea:nonrepetitive} posed in 2002.
  \item \thmref{product-structure} has been used to produce (asymptotically) optimal labelling schemes for planar graphs \cite{dujmovic.esperet.ea:adjacency}.  This (asympotically) resolves a problem of Kannan, Naor, and Rudich posed in 1988 \cite{kannan.naor.ea:implicit-stoc,kannan.naor.ea:implicit}.
  \item \thmref{product-structure} has been used to make significant improvements on the best-known bounds for $p$-centered colourings of planar graphs \cite{debski.felsner.ea:improved}.  This gives the strongest result thus far on a question motivated by the work of Nešetřil and Ossona de Mendez from 2006 \cite{nesetril.ossona:tree,nesetril.ossona:grad} and posed explicity by Dvořák in 2016 \cite{dvorak:question}.
\end{itemize}

The proof of \thmref{product-structure} given by Dujmović \etal\  is based on a similar decomposition of Pilipczuk and Siebertz \cite{pilipczuk.siebertz:polynomial-soda} which is constructive and leads to an $O(n^2)$ time algorithm for finding $H$ and the mapping from $V(G)$ onto $V(H\boxtimes P)$ \cite[Section~10]{dujmovic.joret.ea:planar}. Given the number of applications of \thmref{product-structure} (and that more are likely to be found), it is natural to ask if this running-time can be improved.  In this paper, we provide a faster algorithmic version of \thmref{product-structure}:

\begin{thm}\thmlabel{main}
  For any $n$-vertex planar graph $G$, there exists a graph $H$ of treewidth at most 8 and a path $P$ such that $G$ is a subgraph of $G^+:=H\boxtimes P$.

  Furthermore, there exists an algorithm that, given $G$ as input, runs in $O(n\log n)$ time and produces the graph $H$, the path $P$, and an injective function $\varphi:V(G)\to V(G^+)$ such that, for each edge $vw\in E(G)$,  $\varphi(v)\varphi(w)\in E(G^+)$.
\end{thm}

The remainder of this paper is organized as follows. \Secref{original} reviews the proof of \thmref{product-structure} and the resulting $O(n^2)$ time algorithm.  \Secref{faster} describes the $O(n\log n)$ time algorithm.  \Secref{discussion} discusses some of the implications and generalizations of this work.


\section{The Original Proof/Algorithm}
\seclabel{original}

Throughout this paper we use standard graph theory terminology as used in the textbook by Diestel \cite{diestel:graph}.  Every graph $G$ that we consider is finite, simple, and undirected, and has vertex set denoted by $V(G)$ and edge set denoted by $E(G)$.

Let $T$ be a tree rooted at some node $r$ and, for each node $v$ of $T$, let $P_T(v)$ denote the path in $T$ from $v$ to $r$.  The \emph{$T$-depth} of a node $v$ in $T$ is the length of $P_T(v)$.\footnote{The \emph{length} of a path is equal to the number of edges in the path, which is one less than the number of vertices in the path.}  A path $P$ in $T$ is a \emph{vertical path} if no two nodes of $P$ have the same $T$-depth.  Every node $w$ in $P_T(v)$ is a \emph{$T$-ancestor} of $v$ and $v$ is a \emph{$T$-descendant} of every node $w$ in $P_T(v)$.  Note that $v$ is both a $T$-ancestor and $T$-descendant of itself.  A $T$-ancestor or $T$-descendant $x$ of $v$ is \emph{strict} if $x\neq v$.

For a graph $G$ and a partition $\mathcal{P}$ of $V(G)$, the \emph{quotient graph} $G/\mathcal{P}$ is the graph whose vertices $V(G/\mathcal{P})$ are the sets in $\mathcal{P}$ and in which an edge $XY\in E(G/\mathcal{P})$ if and only if there exists $x\in X$ and $y\in Y$ with $xy\in E(G)$.  Dujmović \etal\  \cite{dujmovic.joret.ea:planar} prove \thmref{product-structure} by first adding edges to a planar graph $G_0$ to complete it to a triangulation $G$, computing a breadth-first spanning tree $T$ of $G$ and then applying the following result to $G$ and $T$:

\begin{thm}\thmlabel{triangulation-partition}
  For any $n$-vertex triangulation $T$ and any spanning tree $T$ of $G$, there exists a partition $\mathcal{P}$ of $V(G)$ such that each $P\in\mathcal{P}$ induces a vertical path in $T$ and the quotient graph $H:=G/\mathcal{P}$ has treewidth at most $8$.
\end{thm}

Deriving \thmref{product-structure} from \thmref{triangulation-partition} is just a matter of checking definitions.  The graph $H$ in \thmref{product-structure} is the same graph $H$ in \thmref{triangulation-partition}. The path $P$ in \thmref{product-structure} is simply the path $0,1,2,\ldots,h$ where $h$ is the maximum depth of any node in $T$.  Each vertex $v\in V(G)$ maps to the node $\varphi(v):=(X,y)$ where $X$ is the set in $\mathcal{P}$ that contains $v$ and $y$ is the depth of $v$ in $T$.  It is straightforward to check (using the definition of $\boxtimes$ and the fact that $T$ is a breadth-first search tree) that for any  edge $vw\in E(G)$, $\varphi(v)\varphi(w)\in E(H\boxtimes P)$.

Therefore, we will focus on giving a fast algorithm for \thmref{triangulation-partition}, from which we immediately obtain \thmref{main}.  We begin by describing the proof of Dujmović \etal\  \cite{dujmovic.joret.ea:planar}, which is inductive, and leads naturally to a recursive algorithm.  Refer to \figref{tripod}. The algorithm is initialized with a breadth-first-search tree $T$ of the triangulation $G$.  Each recursive invocation of the algorithm is given as input:
\begin{compactenum}
  \item A cycle $F$ in $G$.

  The subgraph of $G$ induced by the vertices of $F$ and the vertices of $G$ in the interior of $F$ is a near-triangulation, $N$.  The following are preconditions on the cycle $F$:
  \begin{compactenum}[(P1)]
    \item The root $r$ of $T$ is not in the interior of $F$, i.e., $r\not\in\ V(N)\setminus V(F)$.
    \item For every vertex $v\in V(N)\setminus V(F)$, and every $T$-descendant $w$ of $v$, $w\in V(N)\setminus V(F)$.
    \item Prior to this recursive invocation, every vertex of $F$ is already included in some part of the partition $\mathcal{P}$ and no vertex in $V(N)\setminus V(F)$ is included in any part of $\mathcal{P}$.
  \end{compactenum}
  \item Three edges $e_1$, $e_2$, and $e_3$ of $F$ that we will call \emph{portals}.

  Removing $e_1$, $e_2$ and $e_3$ from $F$ splits $F$ into three non-empty paths $P_1$, $P_2$, and $P_3$ where, for each $i\in\{1,2,3\}$, neither endpoint of $e_i$ is included in $P_i$.  The portals satisfy the following precondition:
  \begin{compactenum}[(P1)]\setcounter{enumii}{3}
    \item For each $i\in\{1,\ldots,3\}$, $V(P_i)$ is contained in the union of at most two elements of $\mathcal{P}$.
  \end{compactenum}

\end{compactenum}

\begin{figure}
  \begin{center}
    \includegraphics{figs/sperner-1} \\[1ex]
    \includegraphics{figs/sperner-2} \\[1ex]
    \includegraphics{figs/sperner-explode-2}
  \end{center}
  \caption{A single recursive step from Dujmović \etal\  \cite{dujmovic.joret.ea:planar}.}
  \figlabel{tripod}
\end{figure}

By the time the recursive invocation terminates, each vertex of $N-V(F)$ is included in some part of the partition $\mathcal{P}$. Let $f$ denote the number of inner triangular faces of $N$.  The base case occurs when $f=1$ so $N$ consists of a single triangle ($F$).  In this case (P3) implies that each vertex of $N$ is already included in $\mathcal{P}$ and there is nothing to do so the algorithm returns immediately.

If $f>1$, the paths $P_1$, $P_2$, and $P_3$, along with the breadth-first search tree $T$ are used to partition the vertices of $N$ into three colour classes, as follows.  Each vertex $v\in P_i$ has colour $c(v)=i$.  For each vertex $v\in V(N)\setminus V(F)$, (P1) implies that $P_T(v)$ contains some first vertex $v_F$ of $F$.  The vertex $v$ is assigned the colour $c(v)=c(v_F)$.

By Sperner's Lemma, $N$ contains a triangular face $\tau=x_1x_2x_3$ that is \emph{trichromatic}, i.e., $c(x_i)=i$ for each $i\in\{1,2,3\}$. (Note that $0$, $1$, $2$, or $3$ vertices of $\tau$ may be in $V(F)$.)  The edges of $F$, $\tau$, and the paths in $T$ from each $x_i$ to the first vertex of $P_i$ define a graph $M$ with at most 4 interior faces, one of which is $\tau$.  Each of the other (at most three) interior faces does not contain $x_i$ for some $i\in\{1,2,3\}$. For each $i\in\{1,2,3\}$, we let $Q_i$ denote the face that does not contain $x_i$. Observe that, for each $i\in\{1,2,3\}$, $Q_i$ contains no vertex of $P_i$.

For each $i\in\{1,2,3\}$, let $Z_i$ be the path, in $T$, from $x_i$ up to, but not including the first vertex in $P_i$.  Note that $Z_i$ may be empty, which occurs when $x_i$ is a vertex of $P_i$.  Let $Y:=V(Z_1)\cup V(Z_2)\cup V(Z_3)$.  The algorithm adds $V(Z_1)$, $V(Z_2)$ and $V(Z_3)$ to the partition $\mathcal{P}$ and then recurses on each of $Q_1$, $Q_2$, and $Q_3$.

We now argue that $Q_1$ satisifies preconditions (P1)--(P3). The face $Q_1$ is a cycle in $G$ that is contained in the cycle $F$, so $Q_1$ satisifies precondition (P1).  The vertices of $Q_1$ are contained in $V(P_2)\cup V(P_3)\cup Y$.  Therefore every vertex of $Q_1$ is contained in some part of $\mathcal{P}$, so $Q_1$ satisfies precondition (P3).  For each vertex $w$ in the interior of $Q_1$, every $T$-ancestor of $w$ is either in $Y$, $F$, or the exterior of $F$.  Therefore $Q_1$ satisifies precondition (P2).

Next we describe the three portals used when recursing on $Q_1$.
The cycle $Q_1$ contains at least one vertex each from $V(P_2)$ and $V(P_3)$ and therefore also contains the portal $e_1$, which is also used as one of the three portals in the recursive invocation.  If $V(Z_2)\cup V(Z_3)$ is non-empty, then $Q_1$ contains two edges $e_2'$ and $e_3'$ where $e_2'$ has an endpoint in $V(P_3)$ and an endpoint in $V(Z_2)\cup V(Z_3)$ and where $e_3'$ has an endpoint in $V(P_2)$ and an endpoint in $V(Z_2)\cup V(Z_3)$.  In this case, the edges $e_1$, $e_2'$, and $e_3'$ are used as the three portal in the recursive invocation on $Q_1$.  Note that $e_1$, $e_2'$ and $e_3'$ satisfy precondition (P4) since the vertices of $P_1'$---the path from $e_2'$ to $e_3'$ on $Q_1$ that does not contain $e_1$---are contained in the union of $V(Z_2)$ and $V(Z_3)$, which are included in $\mathcal{P}$.

If $(V(Z_2)\cup V(Z_3))$ is empty---because $x_2\in V(P_2)$ and $x_3\in V(P_3)$---then we artifically create two portals $e_2'$ and $e_3'$ for the recursive invocation by taking any two edges of $Q_1$ other than $e_1$. Clearly, this choice of $e_2'$ and $e_3'$ also satisfies precondition (P4).

The recursive invocations on $Q_2$ and $Q_3$ are done similarly, but rotating the values $1,2,3$.  After these three recursive invocations, every vertex in $N-V(F)$ is included in some part of $P$, so the recursive invocation is complete.  Dujmović \etal\  then show that the contraction $H:=G/\mathcal{P}$ has treewidth at most 8. There is no need to repeat their argument here.  Instead we discuss the running time of this algorithm.

Recall that $f$ denotes the number of inner faces in the near-triangulation $N$.  By having each vertex of $G$ store a pointer to its parent in $T$ and storing $G$ using a representation that simultaneously represents $G$ and its dual graph $G^*$, the colouring of the vertices of $N$ can be done in $O(f)$ time and then the inner triangular faces of $N$ can be traversed in $O(f)$ time to find the trichromatic triangle $\tau$. The rest of the work (adding $Z_1$, $Z_2$, and $Z_3$ to $P$ and preparing the recursive invocations on $Q_1$, $Q_2$, and $Q_3$) is also easily implemented in $O(f)$ time, so the running time of the algorithm is given by the recurrence
\[  T(f) \le \begin{cases}
           a & \text{for $f\le 1$} \\
           a\cdot f + T(f_1)+T(f_2)+T(f_3) & \text{for $f\ge 2$}
         \end{cases}
 \]
where $a$ is a sufficiently large constant and, for each $i\in\{1,\ldots,3\}$, $f_i$ is the number of faces of $G$ contained in the interior of $Q_i$.
Note that $f_1+f_2+f_3=f-1$ (since $\tau$ is not contained in $Q_1$, $Q_2$, or $Q_3$).  An easy inductive proof shows that $T(f) \le a\cdot f\cdot (f+1)/2 = O(f^2)$.

The recursive procedure described above is used to prove \thmref{triangulation-partition} as follows.  Given an $n$-vertex triangulation $G$ and a spanning tree $T$ of $G$:
\begin{enumerate}
  \item Define one of the faces incident to the root $r$ of $T$ to be the outer face of $G$ and let $r$, $x$, and $y$ denote the three vertices on the outer face of $G$.
  \item Place $\{r\}$, $\{x\}$, and $\{y\}$ in the partition $\mathcal{P}$ and run the recursive procedure described above on the cycle $F:=rxy$ with the portals $e_1=rx$, $e_2=xy$ and $e_3=yr$.
\end{enumerate}
The first step of this procedure runs in constant time.  The second step requires $\Theta(f^2)=\Theta(n^2)$ time in the worst case.

\section{A Faster Algorithm}
\seclabel{faster}

To obtain a faster algorithm we will create an algorithm (part of) whose running time satisifies the recurrence:
\[  T(f) \le \begin{cases}
         a & \text{for $f\le 1$} \\
         a\cdot(1+\min\{f_1,f_2,f_3\}) + T(f_1)+T(f_2)+T(f_3) & \text{for $f\ge 2$}
       \end{cases}
\]
It is straightforward to show, by induction, that $T(f)\le (a/3)f\log_3(f)=O(a\cdot f\log f)$.  The value of $a$ here depends on the running time of an operation on a certain data structure described below.
% Indeed, the inductive step proceeds as follows:
% \begin{align*}
%   T(f) & \le a\min\{f_1,f_2,f_3\} + T(f_1)+T(f_2)+T(f_3) \\
%    & \le (f-1)/3 + T(f_1)+T(f_2)+T(f_3) &\text{since $f_1+f_2+f_3=f-1$} \\
%    & \le (f-1)/3 + f_1\logT(f_1)+T(f_2)+T(f_3) &\text{since $f_1+f_2+f_3=f-1$} \\

Our algorithm makes use of a data structure that preprocesses a $(n+1)$-vertex tree $T$ with root $r$ whose nodes are all initially uncoloured and that supports the following operations:
\begin{enumerate}
   \item $\textsc{Colour}(v,c)$: Set the colour $c(v)$ of the node $v$ of $T$ to some integer value $c$.

   \item $\textsc{FirstColour}(w)$: Return the colour of the first coloured node on the path from node $w$ to the root of $T$.
\end{enumerate}
% We will deliberately conflate the data structure for $T$ and the spanning tree $T$ of $G$ used by the algorithm.

Gabow and Tarjan \cite{gabow.tarjan:linear} prove the following result:\footnote{In the terminology of Gabow and Tarjan \cite{gabow.tarjan:linear}, $\textsc{Colour}(v,c)$ corresponds to \emph{marking} a node $v$ and $\textsc{FirstColour}(w)$ corresponds to find the \emph{nearest marked ancestor} of $w$.}
\begin{thm}[Gabow and Tarjan (1985)]\thmlabel{data-structure}
 Any rooted $n$-vertex tree $T$ can be processed in $O(n)$ time so that any sequence consisting of a total of $m\ge n$ $\textsc{Colour}(v,c)$ and
 $\textsc{FirstColour}(w)$ operations can be performed in $O(m)$ time.
\end{thm}

In the remainder of this section, we will show how \thmref{data-structure} can be used to achieve the desired running time.  In particular, we will show that the number of calls to $\textsc{FirstColour}(w)$ made in our algorithm is $O(n\log n)$ and the number of calls to $\textsc{Colour}(v,c)$ is $O(n)$.  Since we are only interested in the overall running-time of our algorithm, we will treat each such call as taking $O(1)$ time.

It is worth noting that the algorithm we now describe produces exactly the same partition $\mathcal{P}$ produced by the algorithm of Dujmović \etal\ and therefore $\mathcal{P}$ has all the properties described by Dujmović \etal\.   In particular, the quotient graph $H:=G/\mathcal{P}$ has treewidth at most $8$.

As before, each recursive step takes as input the cycle $F$ and the three portals $e_1$, $e_2$, and $e_3$.  Additionally, the algorithm requires that the vertices of $P_1$, $P_2$ and $P_3$ are coloured with three different colours.  More precisely, there are three distinct integers $c_1$, $c_2$ and $c_3$ such that $c(v)=c_i$ for each $v\in V(P_i)$ and each $i\in\{1,2,3\}$.

 The algorithm searches for the trichromatic triangle $\tau$ beginning from the portals.  Refer to \figref{fast-search}. Step~0 of the search begins with $e_{i,0}=e_i$ and $t_{i,0}$ as the unique triangular inner face of $N$ with $e_i$ on its boundary, for each $i\in\{1,2,3\}$. In Step~$j$ of the search, the algorithm has three triangles $t_{i,j}$ and three edges $e_{i,j}$ where $e_{i,j}$ is an edge of $t_{i,j}$ for each $i\in\{1,2,3\}$.  Using the data structure for $T$, the algorithm checks, for each $i\in\{1,2,3\}$, the colours of $t_{i,j}$'s three vertices by calling $\textsc{FirstColour}(w)$.\footnote{Note that each vertex $w$ of $t_{i,j}$ satisfy the precondition for the argument $w$ of $\textsc{FirstColour}(w)$ since every vertex of $F$ is already coloured and, by (P1), the path $P_{T}(w)$ contains at least one vertex of $F$.}  If $t_{i,j}$ is trichromatic for at least one $i\in\{1,2,3\}$, then the algorithm has found the necessary trichromatic triangle $\tau$ and this step is complete. Otherwise, for each $i\in\{1,2,3\}$, the triangle $t_{i,j}$ contains another bichromatic edge $e_{i,j+1}\neq e_{i,j}$ and this edge bounds another triangular face $t_{i,j+1}\neq t_{i,j}$ of $N$.  The algorithm then continues to Step~$(j+1)$ of the search using the triangles $t_{i,j+1}$ and edges $e_{i,j+1}$ for each $i\in\{1,2,3\}$.  The fact that this algorithm terminates (and would even terminate if the search were limited to any one of the portals) follows from a classic proof of Sperner's Lemma in 2-dimensions.

 \begin{figure}
   \begin{center}
     \includegraphics{figs/sperner-4} \\[1ex]
     \includegraphics{figs/sperner-5}
   \end{center}
   \caption{Searching for the trichromatic triangle $\tau$ beginning at the portals $e_1$, $e_2$, and $e_3$. In this example, $\tau=t_{1,6}$ is found after $k+1=7$ steps.}
   \figlabel{fast-search}
 \end{figure}


Suppose the search for $\tau$ succeeds when $\tau=t_{i,k}$ in Step~$k$.  Thus, for each $i\in\{1,2,3\}$, the algorithm has searched the sequence of triangles $t_{i,0},\ldots,t_{i,k}$.  Each of the shorter subsequences $S_i:=t_{i,0},\ldots,t_{i,k-1}$ consists entirely of bichromatic triangles. Each sequence $S_i$ contains $k$ bichromatic triangles whose vertices are coloured with $\{c_1,c_2,c_3\}\setminus c_i$.

Refer to the second part of \figref{fast-search}. Consider again the graph $M$ with faces $Q_1$, $Q_2$, $Q_3$ and $\tau$. For each $i\in\{1,2,3\}$, each face in $S_i$ is contained in $Q_i$.  Since $f_i$ counts the number of triangular faces of $N$ contained in $Q_i$, this implies that $f_i\ge k$ for each $i\in \{1,2,3\}$.  Therefore, $\min\{f_1,f_2,f_3\}\ge k$.  On the other hand, the search for for $\tau$ took $1+k$ steps, each of which performs three $\textsc{FirstColour}(w)$ queries and therefore the entire search runs in time $O(1+k) \subseteq O(1+\min\{f_1,f_2,f_3\})$.

Next, the algorithm prepares the three subproblems defined by $Q_1$, $Q_2$, and $Q_3$ on which to recurse.  To do this it follows the path, in $T$, from each vertex $x_i$ of $\tau$ to the first vertex of $P_i$, calling $\textsc{Colour}(v,c_4)$ for each vertex $v$ it encounters with any integer $c_4\not\in\{c_1,c_2,c_3\}$.

Finally, in preparing each subproblem $Q_i$ for the recursive invocation, it may be necessary to recolour an already coloured vertex $v$ of $F$ with the colour $c_4$ before making the recursive call and then recolouring $v$ with its original colour once the recursion is complete.  This corresponds to introducing an artificial portal adjacent to an edge of $\tau$ contained in $F$.

\subsection{Running-Time Analysis}

We analyze the running time of the preceding algorithm by analyzing two parts separately.

During each recursive invocation, the algorithm does work to find the trichromatic triangle $\tau$.  The time associated with this is $O(1+k)$ where $k\ge \min\{f_1,f_2,f_3\}$.  As already described above, this leads to a recurrence of the form $T(f) \le O(\min\{f_1,f_2,f_3\}) + T(f_1)+T(f_2)+T(f_3)$ which resolves to $O(f\log f)$.  In the initial call, $f=2n-3$ is the number of inner faces of $G$, so the total running time attributable to this part of the algorithm is $O(n\log n)$.

In addition to this, the algorithm does other work in preparing inputs for recursive calls.  Once $\tau$ is identified, the previously uncoloured vertices of $Y$ are coloured with a call to $\textsc{Colour}(v,c)$.  Since each of the $n$ vertices of $G$ are coloured at most once for the first time, this takes a total of $O(n)$ time.  In addition to this, preparing the subproblem $Q_1$, $Q_2$, and $Q_3$ may require an additional 6 calls to $\textsc{Colour}(v,c)$. However, the number of recursive invocations of the algorithm is exactly $2n-3$, since the triangle $\tau$ identified during the invocation is not included in any of the recursive calls.  Therefore, the total running time attributable to these parts of the algorithm is $O(n)$.  This completes the proof of our main theorem:

\begin{thm}
  There exists an algorithm that, given any $n$-vertex triangulation $T$ and any breadth-first-spanning tree $T$ of $G$, runs in $O(n\log n)$ time and finds a partition $\mathcal{P}$ of $V(G)$ such that each $P\in\mathcal{P}$ induces a vertical path in $T$ and the quotient graph $H:=G/\mathcal{P}$ has treewidth at most $8$.
\end{thm}

%
% \subsection{The Data Structure}
%
% All that remains is to describe the required data structure.
%
%  \begin{lem}
%    There exists a data structure that, given an $n$-vertex rooted tree $T$, uses $O(n\log n)$ space, can be constructed in $O(n\log n)$ time, supports the operation $\textsc{FirstColour}(v)$ in $O(1)$ time and supports $m$ $\textsc{Colour}(e,c)$ operations in $O(n\log n)$ time.
%  \end{lem}
%
%  \begin{proof}
%      We decompose $T$ in a sequence $F_0,\ldots,F_r$ of forests, where $F_0:=T$ and, for each $i\in\{1,\ldots,r\}$, $F_i$ is a spanning subgraph of $F_{i-1}$ and each tree in $F_i$ has size at most $\lceil n/2^i\rceil$.
%
%      To obtain the forest $F_i$ we take each tree $T'$ in $F_{i-1}$
%      a find a node $v$ of $T'$ that has at least $|T'|/2$ $T'$-descendants and such that each child of $v$ has at most $|T'|/2$ $T'$-descendants.  We then remove the edges between $v$ and its children in $T'$.  Doing this for every tree $T'$ in $F_{i-1}$ gives the (refined) forest $F_i$.  This process ends after at most $r:=\lceil\log_2 n\rceil$ iterations, at which point $F_r$ has no edges.
%
%      For each node $x$ of $T$ we store a list of $r+1$ pointers $p_{x,0},\ldots,p_{x,r}$ where $p_{x,i}$ is a pointer to the root of the tree $T_{i,x}$ in $F_i$ that contains $x$.  For each $i\in\{0,\ldots,r\}$, and each tree $T_{i,j}$ in the forest $F_i$ with $|T_{i,j}\ge 2|$, the root of $T_{i,j}$ stores a pointer to the node $v$ that is used to split $T$ into $2$ or more trees in $F_{i+1}$.
%
%      Each node $x$ of $T$ stores two values:
%      \begin{compactenum}
%        \item $c_x$ is the colour (if any) assigned to $x$ by the most recent call to $\textsc{Colour}(x)$.
%        \item $i_x$ is the largest $i\in\{1,\ldots,r\}$ such that the path from $x$ to the root of the tree $T_{i,x}$ in the forest $F_i$ contains a coloured vertex, or $-1$ if there is no coloured vertex on the path from $x$ to the root of $T$.  Note that the value of $i_x$ is non-decreasing on root-to-leaf paths in $T$.
%      \end{compactenum}
%
%      With this preprocessing done, the operations are implemented as follows:
%      \begin{enumerate}
%      \item $\textsc{FirstColour}(x)$:  If $c_x$ is defined, then return $c_x$.   Otherwise, if $i_x$ is not defined, then return $-1$ since the path from $x$ to the root of $T=F_0$ contains no coloured edge.  Otherwise, $i_x$ is defined, so consider the tree $T_{i,x}$ in the forest $F_{i_x}$ that contains $x$.  (The root of $T_{i,x}$ can be accessed with $p_{x,i}$.)  Since $i_x=i$ and $c_x$ is undefined, $|T_{i,x}|\ge 2$.  Therefore, there is a node $v$ of $T_{i,x}$ that was used to split $T_{i,x}$ into two or more trees in $F_{i+1}$.  One of these trees, $T_{i+1,x}$, contains $x$.  No node on the path from $x$ to the root of $T_{i+1,x}$ is coloured, therefore, $v$ must be a $T$-ancestor of $x$ and some edge on the path from $v$ to the root of $T_{i,x}$ is coloured.  Therefore, $i_v\ge i+1$ and we can continue to find $\textsc{FirstColour}(v)$.  This process clearly terminates after at most $r=\lceil\log_2 n\rceil$ rounds and therefore this algorithm runs in $O(\log n)$ time.
%
%      \item $\textsc{Colour}(v,c)$. If $v$ has previously been coloured by a call to $\textsc{Colour}(v,c)$, then there is nothing to do but set $c_v:=c$.  Otherwise, we also set $c_v:=c$ and let $i\in\{0,\ldots,r\}$ be the minimum value such that $v$ is the root of the tree $T_{i,v}$ in $F_i$ that contains $v$.  (The value of $i$ can be found by scanning $p_{v,0},\ldots,p_{v,r}$ for the first $i$ such that $p_{v,i}=v$.)   Since no $T$-descendant of $v$ (including $v$) has been coloured before, $i_w < i$ for each $T$-descendant $w$ of $v$.  We therefore traverse the entire subtree rooted at $v$ and set $i_w:=i$ for each $T$-descendant $w$ of $v$.
%
%      Now, observe that each time $i_w$ is modified for some node $w$ of $T$, it is increased.  It follows that $i_w$ is modified at most $r+2=O(\log n)$ times.  This implies the $O(n\log n)$ upper bound on the running-time of all $\textsc{Colour}(v,c)$. \qedhere
%    \end{enumerate}
%  \end{proof}

\section{Discussion}
\seclabel{discussion}

Another variant of \thmref{triangulation-partition} described by Dujmović \etal\ gives a partition $\mathcal{P}$ of $V(G)$ such that $G/\mathcal{P}$ has treewidth at most 3 and each part $Y\in\mathcal{P}$ is the union of at most 3 vertical paths in $T$.  The algorithm described here also gives an $O(n\log n)$ time algorithm for this variant.

\thmref{product-structure} has been generalized to a number of graph classes including bounded-genus graphs \cite{dujmovic.joret.ea:planar}, apex-minor free graphs \cite{dujmovic.joret.ea:planar}, graphs of bounded-degree from proper-minor closed families \cite{dujmovic.esperet.ea:planar}, and $k$-planar graphs \cite{dujmovic.morin.ea:structure}.  In all cases, these generalizations ultimately involve decomposing the input graph into a number of planar subgraphs and applying \thmref{product-structure} to each of these planar graphs.

In at least two cases, the extra work done in these generalizations can be done in $O(n\log n)$ time.  Combined with \thmref{main}, this gives $O(n\log n)$ time algorithms for the corresponding generalizations of \thmref{product-structure}.

\begin{enumerate}
  \item For graphs $G$ of fixed Euler genus $g$, the result of Dujmović \etal\ \cite{dujmovic.joret.ea:planar} only requires finding a genus-$g$ embedding of $G$, computing a breadth-first spanning tree $T$ of $G$, and computing any spanning-tree $D$ of the dual graph that does not cross edges of $T$.  The two spanning trees $T$ and $D$ can be computed in $O(n)$ time using standard algorithms.  The genus-$g$ embedding of $G$ can be computed in $O(n)$ time using an algorithm of Mohar \cite{mohar:linear}

  \item Given a $k$-plane embedding of a $k$-planar graph $G$, the result of Dujmović, Morin, and Wood \cite{dujmovic.morin.ea:structure} applies \thmref{product-structure} directly to the planar graph obtained by adding a dummy vertex at every point where a pair of edges crosses.

  While the problem of testing $k$-planarity of a graph is NP-complete, even for $k=1$ \cite{grigoriev.bodlaender:algorithms,korzhik.mohar:minimal,urschel.wellens:testing}, there are a number of graph classes that are $k$-planar and in which an embedding can be found easily.  These include (appropriate representations of) map graphs, bounded-degree string graphs, powers of bounded-degree planar graphs, and $k$-nearest-neighbour graphs of points in $\R^2$ \cite[Section~8]{dujmovic.morin.ea:structure}.
\end{enumerate}

The obvious open problem left by our work is that of finding a faster algorithm.  Can the running-time in \thmref{main} be improved to $O(n)$?

\section*{Acknowledgement}

Part of this research was conducted during the Eighth Workshop on Geometry and Graphs, held at the Bellairs Research Institute, January~31--February~7, 2020.  The author is grateful to the other organizers and participants for providing a stimulating research environment.


\bibliographystyle{plainurl}
\bibliography{fasttri}

\end{document}
